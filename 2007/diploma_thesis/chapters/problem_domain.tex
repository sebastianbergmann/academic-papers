\chapter{Problem Domain}
\label{chapter-ProblemDomain}

This chapter gives an introduction to the problem domain of Enterprise Content
Management and Workflow Management.

\section{Enterprise Content Management}

The meaning of the term ''Enterprise Content Management'' can be approached
gradually by looking at the three words that make it up:

\begin{itemize}
\item \textbf{Enterprise} refers to the employees of an enterprise with access
      and editing rights.
\item \textbf{Content} refers to arbitrary content stored in the electronic
      systems of an enterprise.
\item \textbf{Management} refers to a software system for the administration,
      control, and processing of content in an enterprise, both internally
      (in an intranet, for example) and externally (on the internet, for
      example).
\end{itemize}

The Association for Information and Image Management (AIIM) defines
Enterprise Content Management (ECM) as

\begin{quote}
\emph{the technologies used to capture, manage, store, preserve, and deliver
content and documents related to organizational processes. ECM tools and
strategies allow the management of an organization's unstructured information,
wherever that information exists} \cite{AIIM}.
\end{quote}

The usage scenarios of eZ~Publish (see Section~\ref{section-eZPublish}), for
example, range from blogs and personal websites to community portals, company
websites, webshops, business process management, enterprise resource planning,
and document management in both governmental institutions and corporate
environments.

\section{Workflow Management}

The Workflow Management Coalition (WfMC) describes workflow management as

\begin{quote}
\emph{the automation of a business process, in whole or parts, where
documents, information or tasks are passed from one participant to another to
be processed, according to a set of procedural rules} \cite{RA01}.
\end{quote}

Georgakopoulos et. al. define workflow management as a

\begin{quote}
\emph{technology supporting the reengineering of business and information
processes. It involves: (1) defining \emph{workflows}, i.e., describing
those aspects of a process that are relevant to controlling and
coordinating the execution of its tasks [...], and (2) providing for fast
(re)design and (re)implementation of the processes as business needs and
information systems change} \cite{DG95}.
\end{quote}

Workflow management systems are software systems that enable workflow
management.

There are two kinds of workflow management systems: those that are
\emph{activity-based} and those that are \emph{entity-based}. The former have
their focus on the activities that are to be completed throughout the workflow,
the latter focus on entities, such as documents, that are processed by a
workflow \cite{FG02}.

The documentation of the OpenFlow workflow management system \cite{OPENFLOW}
summarizes the purpose of an activity-based workflow management system as
\emph{answering the question ''who must do what, when and how''}:

\begin{itemize}
\item The workflow definition (or \emph{workflow schema}) defines the sequence
      of activities that are to be carried out. It specifies \emph{what}
      should be done and \emph{when} by the definition of activities
      (represented by the \emph{nodes} of a directed graph) and transitions
      (represented by the \emph{edges} of a directed graph).

\item An activity (the \emph{what} part of the issue) represents
      something to be done: reviewing a document, publishing a document,
      placing an order, sending an e-mail, and so on.

\item Transitions define the appropriate sequence of activities for a
      process (the \emph{when} part of the issue).

\item Each activity will have an associated application designed to
      carry out the job: the \emph{how} part.

\item The \emph{who} part is generally the user or system assigned to carry
      out the activity, through its application.
\end{itemize}

Figure~\ref{figure-SampleWorkflow} shows a sample workflow that illustrates
this: The green nodes represent the activities that are to be completed
throughout the workflow. The red edges between the nodes represent the control
flow. Depending on the input that is provided by a user with the appropriate
access rights (blue), the \emph{branch} nodes chooses one of two possible
actions that are encapsulated by so-called \emph{service objects} (yellow).
After one of those two possible actions has been performed, the \emph{merge}
nodes merges the control flow again.

The interaction with the user (to receive input, for instance) is performed
through a so-called \emph{worklist} interface. The software system into
which the workflow management system is integrated queries the workflow
system whether a workflow instance is waiting for input that can be
provided by the current user. The user can then provide the input through
the worklist interface.

\begin{figure}[hbtp]
\begin{tikzpicture}
  \node at ( 0, 6) [what] (Start) {Start};
  \node at ( 0, 4) [what] (Input) {Input};
  \node at ( 0, 2) [what] (Branch) {Branch};
  \node at (-2, 0) [what] (ActionA) {Action};
  \node at ( 2, 0) [what] (ActionB) {Action};
  \node at ( 0,-2) [what] (Merge) {Merge};
  \node at ( 0,-4) [what] (End) {End};
  \node at ( 6, 4) [who] (Role) {Role};
  \node at ( 6, 6) [who] (User) {User};
  \node at (-6, 0) [how] (ServiceObjectA) {Service Object};
  \node at ( 6, 0) [how] (ServiceObjectB) {Service Object};
  \draw [red!50,->] (Start) to (Input);
  \draw [red!50,->] (Input) to (Branch);
  \draw [red!50,->,bend right=45] (Branch) to (ActionA);
  \draw [red!50,->,bend left=45] (Branch) to (ActionB);
  \draw [red!50,->,bend right=45] (ActionA) to (Merge);
  \draw [red!50,->,bend left=45] (ActionB) to (Merge);
  \draw [red!50,->] (Merge) to (End);
  \draw[blue!50,dashed] [->] (Role) to (User);
  \draw[blue!50,dashed] [->] (Role) to (Input);
  \draw[yellow!50,dashed] [->] (ActionA) to (ServiceObjectA);
  \draw[yellow!50,dashed] [->] (ActionB) to (ServiceObjectB);
  \begin{pgfonlayer}{background}
    \filldraw [line width=6mm,join=round,black!10]
              (End.south   -| ActionA.west)
    rectangle (Start.north -| ActionB.east);
  \end{pgfonlayer}
\end{tikzpicture}
\caption[Who must do what when and how?]{\textcolor{blue}{Who} must do \textcolor{green}{what} \textcolor{red}{when} and \textcolor{yellow}{how}?}
\label{figure-SampleWorkflow}
\end{figure}

\clearpage
\section{Summary}

Business enterprises need to reduce the cost of doing business and continually
develop new services and products. Enterprise Content Management, as well as
the related practices of Document Management and Knowlege Management, helps
with storing business-critical content (customer data, documents, etc.) in a
central repository and in a unified way. Business Process Management and
Workflow Management provide the methodologies and software that help with
organizing the processes that operate on this content inside an organization.
