\chapter{Evaluation and Related Work}
\label{chapter-Evaluation}

This chapter evaluates the workflow model (see
Chapter~\ref{chapter-WorkflowModel}) and the software (see
Chapter~\ref{chapter-DesignAndImplementation}) that have been developed as
part of this with regard to the requirements (see
Chapter~\ref{chapter-Requirements}) and compares it related work.

\section{Evaluation}

\subsection{Workflow Model}

The workflow model (see Chapter~\ref{chapter-WorkflowModel}) that is the basis
of the software that has been developed as part of this thesis meets the
requirements set up by eZ~Systems~AS (see
Section~\ref{section-RequirementseZPublishTelemark}). It provides good support
for expressing control flow with its direct support of the basic control flow
patterns (see Section~\ref{subsection-BasicControlFlowPatterns}) and the
workflow patterns for advanced branching and synchronization (see
Section~\ref{subsection-AdvancedBranchingAndSynchronization}). This allows the
expression of operations such as the publishing, removal, and modification of
content objects in eZ~Publish to be expressed through workflows. The support
of sub-workflows allows the decomposition of these workflows into manageable
and reusable parts.

Table~\ref{table-ComparisonWorkflowSystems} compares the expressiveness of
the \texttt{ezcWorkflow} components's \emph{backend language} with regard to
the directly supported workflow patterns to other workflow systems. The
comparison data is partially taken from \cite{BK03}.

\begin{table}[hbtp]
\begin{center}
\begin{tabular}[t]{|p{100pt}|l|l|l|l|l|l|l|l|l|l|l|l|l|}
\hline
\small{\textbf{Workflow Pattern}} &
\begin{sideways}\small{\textbf{ezcWorkflow}}\end{sideways} &
\begin{sideways}\small{\textbf{YAWL}}\end{sideways} &
\begin{sideways}\small{\textbf{eZ Publish 3}}\end{sideways} &
\begin{sideways}\small{\textbf{Galaxia}}\end{sideways} &
\begin{sideways}\small{\textbf{Radicore}}\end{sideways} &
\begin{sideways}\small{\textbf{Visual WorkFlo}}\end{sideways} &
\begin{sideways}\small{\textbf{Verve Workflow}}\end{sideways} &
\begin{sideways}\small{\textbf{Staffware}}\end{sideways} &
\begin{sideways}\small{\textbf{MQSeries Workflow}}\end{sideways} &
\begin{sideways}\small{\textbf{Fort� Conductor}}\end{sideways} &
\begin{sideways}\small{\textbf{HP ChangeEngine}}\end{sideways} &
\begin{sideways}\small{\textbf{Fujitsu i-Flow}}\end{sideways} &
\begin{sideways}\small{\textbf{SAP R/3 Workflow}}\end{sideways} \\
\hline
\small{Sequence} & \small{$\surd$} & \small{$\surd$} & \small{$\surd$} & \small{$\surd$} & \small{$\surd$} & \small{$\surd$} & \small{$\surd$} & \small{$\surd$} & \small{$\surd$} & \small{$\surd$} & \small{$\surd$} & \small{$\surd$} & \small{$\surd$} \\
\hline
\small{Parallel Split} & \small{$\surd$} & \small{$\surd$} & (\small{$\surd$}) & \small{$\surd$} & \small{$\surd$} & \small{$\surd$} & \small{$\surd$} & \small{$\surd$} & \small{$\surd$} & \small{$\surd$} & \small{$\surd$} & \small{$\surd$} & \small{$\surd$} \\
\hline
\small{Synchronization} & \small{$\surd$} & \small{$\surd$} & & \small{$\surd$} & \small{$\surd$} & \small{$\surd$} & \small{$\surd$} & \small{$\surd$} & \small{$\surd$} & \small{$\surd$} & \small{$\surd$} & \small{$\surd$} & \small{$\surd$} \\
\hline
\small{Exclusive Choice} & \small{$\surd$} & \small{$\surd$} & & \small{$\surd$} & \small{$\surd$} & \small{$\surd$} & (\small{$\surd$}) & \small{$\surd$} & (\small{$\surd$}) & \small{$\surd$} & \small{$\surd$} & \small{$\surd$} & \small{$\surd$} \\
\hline
\small{Simple Merge} & \small{$\surd$} & \small{$\surd$} & & (\small{$\surd$}) & \small{$\surd$} & \small{$\surd$} & \small{$\surd$} & \small{$\surd$} & \small{$\surd$} & \small{$\surd$} & \small{$\surd$} & \small{$\surd$} & \small{$\surd$} \\
\hline
\small{Multi-Choice} & \small{$\surd$} & \small{$\surd$} & & & & (\small{$\surd$}) & \small{$\surd$} & (\small{$\surd$}) & \small{$\surd$} & \small{$\surd$} & \small{$\surd$} & (\small{$\surd$}) & (\small{$\surd$}) \\
\hline
\small{Synchronizing Merge} & \small{$\surd$} & \small{$\surd$} & & & \small{$\surd$} & & & \small{$\surd$} & & & & & \\
\hline
\small{Multi-Merge} & & \small{$\surd$} & & & & & \small{$\surd$} & & & \small{$\surd$} & & & \\
\hline
\small{Discriminator} & \small{$\surd$} & \small{$\surd$} & & & & & \small{$\surd$} & & & (\small{$\surd$}) & \small{$\surd$} & & \small{$\surd$} \\
\hline
\small{Arbitrary Cycles} & \small{$\surd$} & \small{$\surd$} & & & \small{$\surd$} & & \small{$\surd$} & \small{$\surd$} & & \small{$\surd$} & \small{$\surd$} & \small{$\surd$} & \\
\hline
\small{Implicit Termination} & \small{$\surd$} & & & & & & & \small{$\surd$} & \small{$\surd$} & & & & \\
\hline
\small{Multiple Instances without Synchronization} & & \small{$\surd$} & & & & \small{$\surd$} & \small{$\surd$} & & & \small{$\surd$} & & \small{$\surd$} & \\
\hline
\small{Multiple Instances with A Priori Design Time Knowledge} & & \small{$\surd$} & & & & \small{$\surd$} & \small{$\surd$} & \small{$\surd$} & \small{$\surd$} & \small{$\surd$} & \small{$\surd$} & \small{$\surd$} & \small{$\surd$} \\
\hline
\small{Multiple Instances with A Priori Runtime Knowledge} & & \small{$\surd$} & & & & & & & & & & & (\small{$\surd$}) \\
\hline
\small{Multiple Instances without A Priori Runtime Knowledge} & & \small{$\surd$} & & & & & & & & & & & \\
\hline
\small{Deferred Choice} & & \small{$\surd$} & & & & & & (\small{$\surd$}) & & & & & \\
\hline
\small{Interleaved Parallel Routing} & & \small{$\surd$} & & & & & & & & & & & \\
\hline
\small{Milestone} & & \small{$\surd$} & & & & & & & & & & & \\
\hline
\small{Cancel Activity} & & \small{$\surd$} & & & & & & \small{$\surd$} & & & & & \small{$\surd$} \\
\hline
\small{Cancel Case} & \small{$\surd$} & \small{$\surd$} & & & & \small{$\surd$} & \small{$\surd$} & & & \small{$\surd$} & \small{$\surd$} & & \small{$\surd$} \\
\hline
\end{tabular}
\caption{Comparison of Workflow Systems}
\label{table-ComparisonWorkflowSystems}
\end{center}
\end{table}

\subsection{Implementation}

The software that has been developed as part of this thesis meets the
requirements set up by eZ~Systems~AS (see
Section~\ref{section-RequirementseZPublishTelemark}). It has been
implemented using version~5 of the PHP programming language and is
customizable and extendable. Its architecture allows the addition and
customization of components for workflow execution, persistence, history,
monitoring, and worklist management, for instance. The backend language
that is understood by its virtual machine can be extended by implementing
new node classes. The data storage for workflow schemas and the persistence
of workflow instances has been abstracted, reference implementations for
relational databases (workflow schemas and persistence) and XML files
(workflow schemas only) are available. The workflow schemas are stored in
such a way that they are versioned, old and new versions of a workflow can
be executed at the same time. The \emph{Workflow Execution API} provides
access to information on the workflow instances that are currently
executing and it is possible to manually control the workflow instances
that are currently executing. Through PHP's native SOAP support, the
\emph{Workflow Execution API} can be exposed as a web service, thus
faciliating a distributed and federated workflow environment where one
workflow on one server can start another workflow on another server, for
instance. A special purpose implementation of the workflow virtual machine,
\texttt{ezcWorkflowTestExecution}, allows for the simulation of workflow
execution for debugging and testing.

\section{Related Work}

\subsection{Research}

\subsubsection{Micro-Workflow}

Manolescu proposes \emph{a new workflow architecture that bridges the gap
between the type of functionality provided by current workflow systems and
the type of workflow functionality required in object-oriented applications}.
In his PhD thesis \cite{DAM01}, he discusses the design and implementation of
Micro-Workflow, an object-oriented framework that is built using this
architecture. One of Manolescu's key findings is that more advanced workflow
features can be added to light-weight workflow core through composition.

\subsubsection{SWAMP}

In his Diploma thesis \cite{TS04}, Schmidt discusses the design and
implementation of the SWAMP workflow system (SuSE Workflow and Management
Platform). The goal of Schmidt's thesis is to replace an inhomogeneous
legacy system with a unified workflow system that is easier to maintain and
that can be easily customized and extended. The motivations behind and the
requirements for the SWAMP workflow system have some similarities with the
needs for the software that has been developed as part of this thesis.

\subsubsection{YAWL}

In her Master thesis \cite{SH05}, Heijens discusses the design and
implementation of YAWL. YAWL is both a workflow language (Yet Another Workflow
Language) and a workflow system. The workflow model of YAWL is formally based
on Petri nets and supports all the workflow patterns with the exception of
\emph{Implicit Termination}.

\subsection{Workflow Systems for PHP}

\subsubsection{eZ Publish 3}

From the workflow patterns that were discussed in
Section~\ref{section-WorkflowPatterns}, the workflow system of eZ~Publish~3
only supports the \emph{Sequence} workflow pattern directly. Through its
\emph{Multiplexer} workflow event, which starts another workflow from within
a workflow, it indirectly supports the \emph{Parallel Split} workflow pattern.

The technical limitations of the workflow system in eZ~Publish~3 (see
Section~\ref{section-RequirementseZPublish3}) are representative for other
content management systems for the PHP platform (and most likely for those for
other platforms as well). The tight integration with the application into
which the workflow system is embedded makes the independent usage of the
workflow system impossible.

\subsubsection{Galaxia Workflow Engine}

The Galaxia Workflow Engine \cite{GF03} is an activity-based workflow engine
for PHP that is loosely based on OpenFlow \cite{OPENFLOW}.

The graphical workflow description language supported by the Galaxia workflow
engine consists of six \emph{activity types}:

\begin{enumerate}
\item \emph{Start} represents the beginning of a workflow.
\item \emph{End} represents the end of a workflow.
\item \emph{Activity} represents an activity that is to be performed.
\item \emph{Switch} represents a point of decision in the workflow and can be
      compared to the \emph{Exclusive Choice} workflow pattern.
\item \emph{Split} is equivalent to the \emph{Parallel Split} workflow
      pattern.
\item \emph{Join} is equivalent to the \emph{Synchronization} workflow
      pattern.
\end{enumerate}

Galaxia does not have an explicit \emph{Simple Merge} construct to merge
the multiple possible threads of a \emph{Switch} construct. Instead, an
\emph{Activity} construct implicitly merges its incoming threads.

\subsubsection{Radicore}

The Radicore toolkit for PHP features an activity-based workflow engine that
is based on Petri nets \cite{TM04}.

It supports

\begin{itemize}
\item Sequential Routing.
\item Parallel Routing through \emph{AND-Split} and \emph{AND-Join} constructs.
\item Conditional Routing through explicit and implicit \emph{OR-Split} and
      \emph{OR-Join} constructs.
\item Iterative Routing using the \emph{OR-Split} construct.
\end{itemize}

These constructs correspond to the \emph{Sequence}, \emph{Parallel Split},
\emph{Synchronization}, \emph{Exclusive Choice}, and \emph{Simple Merge}
workflow patterns (see Section~\ref{section-WorkflowPatterns}).

\section{Summary}

Workflow systems such as Micro-Workflow \cite{DAM01} and YAWL \cite{SH05}, for
instance, that have been implemented as part of academic research often excel
only in the aspect that is specific to the research while neglecting other
aspects that are relevant to a workflow system. One of the goals of this
thesis was to combine aspects such as component-based workflow architecture,
workflow virtual machine, and workflow patterns to create a workflow system
that meets industry requirements.

The existing workflow systems for PHP do not lend themselves well to
customization and extension with regard to the requirements set up by
eZ~Systems~AS. This fact, together with eZ~Systems' requirement for clear
intellectual property, lead to the development of a new workflow engine
instead of starting with an existing one.
