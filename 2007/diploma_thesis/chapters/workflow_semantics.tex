\chapter{Workflow Semantics}
\label{chapter-WorkflowSemantics}

This chapter presents \emph{Petri nets} as a formal way and
\emph{workflow patterns} as a more pragmatic way to define the semantics of a
workflow model.

\section{Petri Nets}

A Petri net is a mathematical representation for discrete distributed systems.
It is a 5-tuple $(S, T, F, M_0, W$), where \cite{JD01}

\begin{itemize}
\item $S$ is a set of \emph{places}.
\item $T$ is a set of \emph{transitions}.
\item $F$ is a set of \emph{arcs} known as a \emph{flow relation}. The set
      $F$ is subject to the constraint that no arc may connect two places
      or two transitions, or more formally:
      $F \subseteq (S \times T) \bigcup (T \times S)$.
\item $M_0 : S \rightarrow \mathbb{N}$ is an \emph{initial marking}, where
      for each place $s \in S$, there are $n_s \in \mathbb{N}$ tokens.
\item $W : F \rightarrow \mathbb{N}$ is a set of \emph{arc weights}, which
      assigns to each arc $f \in F$ some $n \in \mathbb{N}^+$ denoting how
      many tokens are consumed from a place by a transition, or alternatively,
      how many tokens are produced by a transition and put into each place.
\end{itemize}

A place that has an arc to a transition is called an \emph{input place}, a
place that has an arc from a transition is called an \emph{output place}. A
place may contain any number of tokens. A distribution of tokens over the
places of a net is called a \emph{marking}. When a transition is activated
(or \emph{fired}), it consumes the tokens from its input places, performs
some form of processing, and places a specified number of tokens in its
output places. These three steps are performed atomically. The activation
of transitions is non-deterministic. This makes Petri nets well suited for
the modelling of concurrent behaviour in distributed systems.

Petri nets have been proposed for modelling workflows by van der Aalst
\cite{WA96}, for instance, because they provide \emph{formal semantics
despite the graphical nature} and an \emph{abundance of analysis
techniques} exists. Figures \ref{figure-Causality} to \ref{figure-ORJoin}
show how the workflow primitives of the workflow reference model
\cite{WfMC95} can be expressed using Petri nets.

\begin{figure}[htb]
  \begin{minipage}{0.45\textwidth}
    \begin{center}
      \begin{tikzpicture}
        \node at (0,0) [place,tokens=1] (a) {};
        \node at (1,0) [transition]     (b) {$t_{11}$};
        \node at (2,0) [place]          (c) {};
        \node at (3,0) [transition]     (d) {$t_{12}$};
        \node at (4,0) [place]          (e) {};

        \draw [->] (a) -- (b);
        \draw [->] (b) -- (c);
        \draw [->] (c) -- (d);
        \draw [->] (d) -- (e);
      \end{tikzpicture}
      \caption[The \emph{Causality} workflow primitive]{Causality}
      \label{figure-Causality}
    \end{center}
  \end{minipage}
  \hfill
  \begin{minipage}{0.45\textwidth}
    \begin{center}
      \begin{tikzpicture}
        \node at (0,0) [place,tokens=1] (a) {};
        \node at (2,0) [transition]     (b) {$t_{21}$};
        \node at (4,0) [place]          (c) {};
        \node at (2,2) [transition]     (d) {$t_{22}$};

        \draw [->] (a) -- (b);
        \draw [->] (b) -- (c);
        \draw [->] (c) -- (d);
        \draw [->] (d) -- (a);
      \end{tikzpicture}
      \caption[The \emph{Iteration} workflow primitive]{Iteration}
      \label{figure-Iteration}
    \end{center}
  \end{minipage}
\end{figure}

\begin{figure}[htb]
  \begin{minipage}{0.45\textwidth}
    \begin{center}
      \begin{tikzpicture}
        \node at (0,1) [place,tokens=1] (a) {};
        \node at (2,1) [transition]     (b) {$t_{3}$};
        \node at (4,2) [place]          (c) {};
        \node at (4,0) [place]          (d) {};

        \draw [->] (a) -- (b);
        \draw [->] (b) -- (c);
        \draw [->] (b) -- (d);
      \end{tikzpicture}
      \caption[The \emph{AND-Split} workflow primitive]{AND-Split}
      \label{figure-ANDSplit}
    \end{center}
  \end{minipage}
  \hfill
  \begin{minipage}{0.45\textwidth}
    \begin{center}
      \begin{tikzpicture}
        \node at (0,2) [place,tokens=1] (a) {};
        \node at (0,0) [place,tokens=1] (b) {};
        \node at (2,1) [transition]     (c) {$t_{4}$};
        \node at (4,1) [place]          (d) {};

        \draw [->] (a) -- (c);
        \draw [->] (b) -- (c);
        \draw [->] (c) -- (d);
      \end{tikzpicture}
      \caption[The \emph{AND-Join} workflow primitive]{AND-Join}
      \label{figure-ANDJoin}
    \end{center}
  \end{minipage}
\end{figure}

\begin{figure}[htb]
  \begin{minipage}{0.45\textwidth}
    \begin{center}
      \begin{tikzpicture}
        \node at (0,1) [place,tokens=1] (a) {};
        \node at (2,2) [transition]     (b) {$t_{51}$};
        \node at (2,0) [transition]     (c) {$t_{52}$};
        \node at (4,2) [place]          (d) {};
        \node at (4,0) [place]          (e) {};

        \draw [->] (a) -- (b);
        \draw [->] (a) -- (c);
        \draw [->] (b) -- (d);
        \draw [->] (c) -- (e);
      \end{tikzpicture}
      \caption[The \emph{OR-Split} workflow primitive]{OR-Split}
      \label{figure-ORSplit}
    \end{center}
  \end{minipage}
  \hfill
  \begin{minipage}{0.45\textwidth}
    \begin{center}
      \begin{tikzpicture}
        \node at (0,2) [place,tokens=1] (a) {};
        \node at (0,0) [place]          (b) {};
        \node at (2,2) [transition]     (c) {$t_{61}$};
        \node at (2,0) [transition]     (d) {$t_{62}$};
        \node at (4,1) [place]          (e) {};

        \draw [->] (a) -- (c);
        \draw [->] (b) -- (d);
        \draw [->] (c) -- (e);
        \draw [->] (d) -- (e);
      \end{tikzpicture}
      \caption[The \emph{OR-Join} workflow primitive]{OR-Join}
      \label{figure-ORJoin}
    \end{center}
  \end{minipage}
\end{figure}

Finite Automata are another formalism that can be used to describe workflows \cite{ARK03}.

\section{Workflow Patterns}
\label{section-WorkflowPatterns}

In Chapter 3 of his PhD thesis \cite{BK03}, Kiepuszewski lists
\emph{requirements for workflow languages through workflow patterns}.

Much like the software design patterns \cite{GoF94}, these workflow patterns
describe recurring solutions to common problems. They are relevant to both the
implementor and the user of a workflow management system. The former uses the
workflow patterns as a common vocabulary for workflow description language
constructs and to define the semantics of a workflow model (see
Chapter~\ref{chapter-WorkflowModel}) whereas the latter uses them as a guide while
formulating his workflow in the workflow system's description language. The
workflow patterns also faciliate the comparison with other workflow systems
with regard to expressiveness and power.
 
In Chapter 4 of his PhD thesis \cite{BK03}, Kiepuszewski maps the workflow
patterns that he identified to Petri nets to provide a formal foundation for
this more pragmatic approach to defining workflow semantics.

In this section we discuss the subset of the workflow patterns identified by
Kiepuszewski that is directly supported by the software that has been
implemented as part of this thesis.

\subsection{Basic Control Flow Patterns}
\label{subsection-BasicControlFlowPatterns}

The workflow patterns for basic control flow \emph{capture elementary aspects
of process control} and \emph{closely match the definitions of elementary
control flow concepts provided by the Workflow Management Coalition in
\cite{WfMC95,WfMC99}}.

\subsubsection{Sequence}

The \emph{Sequence} workflow pattern represents linear execution of workflow
steps: one action of a workflow is activated unconditionally (for example
\emph{B} in Figure \ref{figure-Sequence}) after another (for example \emph{A}
in Figure \ref{figure-Sequence}) finished executing.

\begin{figure}[hbt]
\begin{center}
\begin{tikzpicture}
  \node at ( 0, 0) [node] (A) {A};
  \node at ( 2, 0) [node] (B) {B};
  \draw [->] (A) to (B);
\end{tikzpicture}
\end{center}
\caption[The \emph{Sequence} workflow pattern]{Sequence}
\label{figure-Sequence}
\end{figure}

\textbf{Use Case Example:} After an order is placed, the credit card specified by the
customer is charged. 

\subsubsection{Parallel Split (AND-Split)}

The \emph{Parallel Split} workflow pattern divides one thread of execution
(for example the one that activates \emph{A} in Figure
\ref{figure-ParallelSplit}) unconditionally into multiple parallel threads of
execution (for example the ones that start in \emph{B}, \emph{C}, and \emph{D}
in Figure \ref{figure-ParallelSplit}).

\begin{figure}[htb]
  \begin{minipage}{0.45\textwidth}
    \begin{center}
      \begin{tikzpicture}
        \node at ( 0, 0) [node] (A) {A};
        \node at ( 2, 2) [node] (B) {B};
        \node at ( 2, 0) [node] (C) {C};
        \node at ( 2,-2) [node] (D) {D};
        \draw [->,bend left=45] (A) to (B);
        \draw [->] (A) to (C);
        \draw [->,bend right=45] (A) to (D);
      \end{tikzpicture}
      \caption[The \emph{Parallel Split} workflow pattern]{Parallel Split}
      \label{figure-ParallelSplit}
    \end{center}
  \end{minipage}
  \hfill
  \begin{minipage}{0.45\textwidth}
    \begin{center}
      \begin{tikzpicture}
        \node at ( 0, 2) [node] (B) {B};
        \node at ( 0, 0) [node] (C) {C};
        \node at ( 0,-2) [node] (D) {D};
        \node at ( 2, 0) [node] (E) {E};
        \draw [->,bend left=45] (B) to (E);
        \draw [->] (C) to (E);
        \draw [->,bend right=45] (D) to (E);
      \end{tikzpicture}
      \caption[The \emph{Synchronization} workflow pattern]{Synchronization}
      \label{figure-Synchronization}
    \end{center}
  \end{minipage}
\end{figure}

\textbf{Use Case Example:} After the credit card specified by the customer has been
successfully charged, the activities of sending a confirmation email and
starting the shipping process can be executed in parallel.

\subsubsection{Synchronization (AND-Join)}

The \emph{Synchronization} workflow pattern synchronizes multiple parallel
threads of execution (for example the ones that end in \emph{B}, \emph{C},
and \emph{D} in Figure \ref{figure-Synchronization}) into a single thread of
execution (for example the one that starts in \emph{E} in Figure
\ref{figure-Synchronization}).

Workflow execution continues once all threads of execution that are to be
synchronized have finished executing (exactly once).

\textbf{Use Case Example:} After the confirmation email has been sent and
the shipping process has been completed, the order can be archived.

The workflow patterns that have been discussed so far handle the
\emph{unconditional routing} of control flow. We will now take a look at the
workflow patterns for \emph{conditional routing}.

\subsubsection{Exclusive Choice (XOR-Split)}

The \emph{Exclusive Choice} workflow pattern defines multiple possible paths
(for example the ones that start in \emph{B}, \emph{C}, and \emph{D} in Figure
\ref{figure-ExclusiveChoice}) for the workflow of which exactly one is chosen
(for example the one that starts in \emph{C} in Figure
\ref{figure-ExclusiveChoice}).

\begin{figure}[htb]
  \begin{minipage}{0.45\textwidth}
    \begin{center}
      \begin{tikzpicture}
        \node at ( 0, 0) [node] (A) {A};
        \node at ( 2, 2) [node] (B) {B};
        \node at ( 2, 0) [node] (C) {C};
        \node at ( 2,-2) [node] (D) {D};
        \draw[dashed] [->,bend left=45] (A) to (B);
        \draw [->] (A) to (C);
        \draw[dashed] [->,bend right=45] (A) to (D);
      \end{tikzpicture}
      \caption[The \emph{Exclusive Choice} workflow pattern]{Exclusive Choice}
      \label{figure-ExclusiveChoice}
    \end{center}
  \end{minipage}
  \hfill
  \begin{minipage}{0.45\textwidth}
    \begin{center}
      \begin{tikzpicture}
        \node at ( 0, 2) [node] (B) {B};
        \node at ( 0, 0) [node] (C) {C};
        \node at ( 0,-2) [node] (D) {D};
        \node at ( 2, 0) [node] (E) {E};
        \draw[dashed] [->,bend left=45] (B) to (E);
        \draw [->] (C) to (E);
        \draw[dashed] [->,bend right=45] (D) to (E);
      \end{tikzpicture}
      \caption[The \emph{Simple Merge} workflow pattern]{Simple Merge}
      \label{figure-SimpleMerge}
    \end{center}
  \end{minipage}
\end{figure}

\textbf{Use Case Example:} After an order has been received, the payment can
be performed by credit card or bank transfer.

\subsubsection{Simple Merge (XOR-Join)}

The \emph{Simple Merge} workflow pattern is to be used to merge the possible
paths that are defined by a preceding \emph{Exclusive Choice}. It is assumed
that of these possible paths exactly one is taken (for example \emph{C} in
Figure \ref{figure-SimpleMerge}) and no synchronization takes place.

\textbf{Use Case Example:} After the payment has been performed by either
credit card or bank transfer, the order can be processed further.

\subsection{Advanced Branching and Synchronization}
\label{subsection-AdvancedBranchingAndSynchronization}

The workflow patterns for advanced branching and synchronization \emph{do
not have straightforward support in most [of the] workflow engines [that
Kiepuszewski evaluated]} (see Table~\ref{table-ComparisonWorkflowSystems}).
\emph{Nevertheless, they are quite common in real-life business scenarios}.

\subsubsection{Multi-Choice (OR-Split)}

The \emph{Multi-Choice} workflow pattern defines multiple possible paths
(for example the ones that start in \emph{B}, \emph{C}, and \emph{D} in Figure
\ref{figure-MultiChoice}) for the workflow of which one or more are chosen
(for example the ones that start in \emph{B} and \emph{D} in Figure
\ref{figure-MultiChoice}). It is a generalization of the Parallel Split and
Exclusive Choice workflow patterns.

\begin{figure}[htb]
  \begin{minipage}{0.45\textwidth}
    \begin{center}
      \begin{tikzpicture}
        \node at ( 0, 0) [node] (A) {A};
        \node at ( 2, 2) [node] (B) {B};
        \node at ( 2, 0) [node] (C) {C};
        \node at ( 2,-2) [node] (D) {D};
        \draw [->,bend left=45] (A) to (B);
        \draw[dashed] [->] (A) to (C);
        \draw [->,bend right=45] (A) to (D);
      \end{tikzpicture}
      \caption[The \emph{Multi-Choice} workflow pattern]{Multi-Choice}
      \label{figure-MultiChoice}
    \end{center}
  \end{minipage}
  \hfill
  \begin{minipage}{0.45\textwidth}
    \begin{center}
      \begin{tikzpicture}
        \node at ( 0, 2) [node] (B) {B};
        \node at ( 0, 0) [node] (C) {C};
        \node at ( 0,-2) [node] (D) {D};
        \node at ( 2, 0) [node] (E) {E};
        \draw [->,bend left=45] (B) to (E);
        \draw[dashed] [->] (C) to (E);
        \draw [->,bend right=45] (D) to (E);
      \end{tikzpicture}
      \caption[The \emph{Synchronizing Merge} workflow pattern]{Synchronizing Merge}
      \label{figure-SynchronizingMerge}
    \end{center}
  \end{minipage}
\end{figure}

\subsubsection{Synchronizing Merge}

The \emph{Synchronizing Merge} workflow pattern is to be used to synchronize
multiple parallel threads of execution that are activated by a preceding
\emph{Multi-Choice} (for example the ones that end in \emph{B} and \emph{D}
in Figure \ref{figure-SynchronizingMerge}).

\subsubsection{Discriminator}

The \emph{Discriminator} workflow pattern can be applied when the assumption
we made for the \emph{Simple Merge} workflow pattern does not hold. It can
deal with merge situations where multiple incoming branches may run in
parallel.

It activates its outgoing node after being activated by the first incoming
branch and then waits for all remaining branches to complete before it resets
itself. After the reset the \emph{Discriminator} can be triggered again.

\textbf{Use Case Example:} To improve response time, an action is delegated to
several distributed servers. The first response proceeds the flow, the other
responses are ignored.

\subsection{Structural Patterns}
\label{subsection-StructuralPatterns}

The structural workflow patterns deal with restrictions that different workflow
models can impose.

\subsubsection{Arbitrary Cycles}

A common restriction workflow models impose is that arbitrary cycles, ie. one
or more activities are done repeatedly, are not supported. As an alternative,
special loop constructs that mark the start and end point of a structured cycle
are offered.

\subsubsection{Implicit Termination}

The execution of the workflow is (successfully) terminated when there are no
activated activities left and no other activity can be activated. This implicit
termination of workflow execution can be used in addition to explicit end
activities.

\subsection{Cancellation Patterns}
\label{subsection-CancellationPatterns}

\subsubsection{Cancel Case}

The execution of a workflow instance is cancelled.

\textbf{Use Case Example:} An order is cancelled.

\section{Summary}

This chapter presented Petri nets as a general, well understood and well
researched, theory for concurrency and the workflow patterns as an pragmatic
approach to describe the semantics of workflow routing constructs.

We will use the workflow patterns in Chapter~\ref{chapter-Requirements} during
the requirements analysis and in Chapter~\ref{chapter-WorkflowModel} to define
the semantics of the workflow model for the software that has been developed as
part of this thesis.
