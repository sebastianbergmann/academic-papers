\chapter{Conclusion and Future Work}
\label{chapter-Conclusion}

\section{Conclusion}

This thesis reviewed previous research such as \cite{BK03,DAM01,PM99,SF04}
and combined it for the first time in an effort to design and implement a
workflow system that meets industry requirements.

The pragmatic approach to describe the semantics of workflow routing constructs
through \emph{Workflow Patterns} \cite{BK03} provides a good foundation for the
\emph{Backend Language} of a \emph{Workflow Virtual Machine} \cite{SF04} that
executes workflow definitions represented through object graphs in a
component-based workflow architecture \cite{DAM01}.

The software that has been developed as part of this thesis is a contribution
to the PHP community. It provides an extendable framework to define workflows
and a virtual machine for the execution of these definitions that can be
embedded into a PHP application, thus extending it with workflow capabilities.
This workflow system can be customized and extended through the composition of
components, its workflow model can be customized and extended through the
classes that define the control flow constructs. It is neither bound to a
specific application into which it is embedded nor to a specific workflow
description language, thus providing more degrees of freedom with regard to
use -- and re-use -- of the workflow engine.

\section{Future Work}

\subsection{Analysis and Verification of Workflows}

The current implementation of the software that has been developed as part of
this thesis has basic support for the analysis and verification of workflow
specifications. Future versions of the software can implement more advanced
verification tools based upon the \emph{abundance of analysis techniques} that
exists for Petri nets \cite{WA96}.

\subsection{Workflow Model}

The workflow model can be extended, for instance, with support for more
workflow patterns, by adding the respective node types.

\subsection{Aspect-Oriented Programming}

Aspect-Oriented Programming \emph{allow[s] programming by making quantified
programmatic assertions over programs written by programmers oblivious to
such assertions} \cite{RF00}. These assertions make \emph{quantified
statements about which code is to execute in which circumstances}.

Section 2 of \cite{SB06} presents an overview of the various implementations
of AOP for PHP that support AOP by extending the base programming language.
The combination of Graph-Oriented Programming with Aspect-Oriented Programming
would add yet another possibility to faciliate AOP with the PHP platform, but
without the need to change or extend the base programming language. The
workflow model discussed in Chapter~\ref{chapter-WorkflowModel} would serve as
the Joinpoint Model of an AOP system that can be implemented as an additional
component for the software that was presented in
Chapter~\ref{chapter-DesignAndImplementation}. Pointcuts such as \emph{node of
type X is executed} could then be used to express when additional code is to
be run during workflow execution. Compared to the implicit callgraph structure
on which language-level AOP systems operate, the explicit graph structure of
workflows makes the idea of AOP intuitively clear.

This combination of Aspect-Oriented Programming with Graph-Oriented Programming
could then be compared to Aspect-Oriented Programming in general as well as
to Adaptive Programming which Lieberherr describes as \emph{the special case of
Aspect-Oriented Programming (AOP) where some of the building blocks are
expressible in terms of graphs} \cite{KL97}.

\subsection{Compilation of Workflows}

Model-Driven Architecture (MDA) \emph{separates business and application logic
from underlying platform technology} \cite{JM01} and supports the Model-Driven
Engineering (MDE) of software systems. It offers a \emph{promising approach to
address the inability of third-generation languages to alleviate the complexity
of platforms and express domain concepts effectively} \cite{DS06}.

In this context the possibility could be evaluated whether the software that
has been developed as part of this thesis can be extended with a code
generator component that can compile a workflow specification into a
ready-to-use application.
