\addcontentsline{toc}{chapter}{Introduction}
\chapter*{Introduction}

\section*{Problem Statement and Goal}

This thesis has the design and implementation of a workflow engine as its goal.
This goal has motivations from both academia and industry that were represented
by the two supervising institutions, the Institute of Computer Science of the
University of Bonn, Germany and eZ~Systems~AS, respectively.

The topic of this diploma thesis was set up by eZ~Systems~AS in Skien,
Norway. The company is the creator of eZ~Publish, an Open Source Enterprise
Content Management System, and eZ~Components, a components library for PHP~5.
As we will see in Chapter~\ref{chapter-Requirements}, eZ~Systems~AS is in need
of a flexible and reusable workflow engine component, written in the PHP
programming language, that can be used in the development of the next version
of their eZ~Publish~ECMS. Of academic interest is how research such as
\cite{BK03,DAM01,PM99,SF04} can be put to use for the design and
implementation of such a software component.

The goal of this thesis is therefore to \emph{review the relevant literature},
to \emph{find a suitable workflow model} as the foundation for the
\emph{design and implementation of a workflow engine}, and to \emph{evaluate
the resulting software component} with regard to the industry requirements
set up by eZ~Systems~AS.

\section*{Structure}

Chapter~\ref{chapter-ProblemDomain} gives an introduction to the problem domain
of Enterprise Content Management and Workflow Management.
Chapter~\ref{chapter-WorkflowSemantics} presents Petri nets as a formal way and
workflow patterns as a more pragmatic way to define the semantics of a workflow
model.

Chapter~\ref{chapter-Technology} gives an introduction to the technology stack
(PHP, eZ~Publish, eZ~Components) that is relevant to and used by the software
that has been implemented as part of this thesis.
Chapter~\ref{chapter-Requirements} discusses the requirements that lead to the
development of this software.

Chapter~\ref{chapter-WorkflowModel} presents the semantics and syntax of the
workflow model that is the foundation for the software.
Chapter~\ref{chapter-DesignAndImplementation} discusses the design and
implementation of the software.

This paper concludes with an evaluation of the software
(Chapter~\ref{chapter-Evaluation}), a comparison to related work, and an
outlook on future work (Chapter~\ref{chapter-Conclusion}).

\section*{Acknowledgements}
\thispagestyle{empty}

I would like to thank Prof. Dr. Armin B. Cremers for making it possible that
I could do my thesis in cooperation with eZ~Systems~AS and Dr. Stefan
L�ttringhaus-Kappel for being my thesis advisor. I would like to thank
everyone at eZ~Systems~AS for the great time I had in Norway.

Finally, I would like to express my appreciation for the people involved in
the development of the free software that is discussed in this paper (PHP,
eZ~Publish, eZ~Components) and was used to typeset (\LaTeX, Dia, Doxygen,
GraphViz, KOMA-Script, PGF, Ti\emph{k}Z) this paper.
