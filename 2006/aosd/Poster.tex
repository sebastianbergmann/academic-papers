\documentclass[landscape,a0]{a0poster}

\usepackage{epsf,psfig,pstricks,multicol,pst-grad,color}
\usepackage{times}
\usepackage{listings}

\def\RGBtext#1#2#3#4{\special{ps::[begin]currentrgbcolor   #1 255 div #2 255 div  #3 255 div setrgbcolor}%
                     #4\special{ps::[begin]setrgbcolor}\special{ps::[begin]}}



%% Changes in a0poster.cls:

\ifanullb
   \setlength{\paperwidth}{119cm}
   \setlength{\paperheight}{88cm}
   \setlength{\textwidth}{116cm}
   \setlength{\textheight}{88cm}
\else\ifanull
        \setlength{\paperwidth}{118.82cm}
        \setlength{\paperheight}{83.96cm}
        \setlength{\textwidth}{117.82cm}
        \setlength{\textheight}{82.96cm}
     \else\ifaeins
             \setlength{\paperwidth}{83.96cm}
             \setlength{\paperheight}{59.4cm}
             \setlength{\textwidth}{83.96cm}
             \setlength{\textheight}{58.4cm}
          \else\ifazwei
                  \setlength{\paperwidth}{59.4cm}
                  \setlength{\paperheight}{41.98cm}
                  \setlength{\textwidth}{58.4cm}
                  \setlength{\textheight}{40.98cm}
               \else\ifadrei
                       \setlength{\paperwidth}{41.98cm}
                       \setlength{\paperheight}{29.7cm}
                       \setlength{\textwidth}{40.98cm}
                       \setlength{\textheight}{28.7cm}
                    \else\relax
                    \fi
               \fi
          \fi
     \fi
\fi
\setlength{\topmargin}{-10.7 mm}  % -1in +1.47cm
\setlength{\oddsidemargin}{-21.4 mm} % -1in +0.4cm

\ifportrait
   \newdimen\tausch
   \setlength{\tausch}{\paperwidth}
   \setlength{\paperwidth}{\paperheight}
   \setlength{\paperheight}{\tausch}
   \setlength{\tausch}{\textwidth}
   \setlength{\textwidth}{\textheight}
   \setlength{\textheight}{\tausch}
\else\relax
\fi


%% Additional commands:
\makeatletter
\renewcommand{\section}{\@startsection {section}{1}{\z@}%
  {-3.5ex \@plus -1ex \@minus -.2ex}%
  {2.3ex \@plus.2ex}%
  {\reset@font\Large\bfseries\sffamily}}
\renewcommand{\subsection}{\@startsection{subsection}{2}{\z@}%
  {-3.25ex\@plus -1ex \@minus -.2ex}%
  {1.5ex \@plus .2ex}%
  {\reset@font\large\bfseries\sffamily}}
\makeatother

%%%%%%%%%%%%%%%%%%%%%%%%%%%%%%%%%%%%%%%%%%%%%%%%%%%%
%%%               Hintergrund                    %%%
%%%%%%%%%%%%%%%%%%%%%%%%%%%%%%%%%%%%%%%%%%%%%%%%%%%%
\newcommand{\background}[3]{\newrgbcolor{cgradbegin}{#1}
  \newrgbcolor{cgradend}{#2} 
  \psframe[fillstyle=gradient,gradend=cgradend,
  gradbegin=cgradbegin,gradmidpoint=#3](0.,0.)(1.\textwidth,-1.\textheight)}

%%%%%%%%%%%%%%%%%%%%%%%%%%%%%%%%%%%%%%%%%%%%%%%%%%%%
%%%                Header                        %%%
%%%%%%%%%%%%%%%%%%%%%%%%%%%%%%%%%%%%%%%%%%%%%%%%%%%%
\newenvironment{header}[1][1. 1. 1.] {
    \vspace{2em}

    \begin{center}
    \newrgbcolor{hcolor}{#1}
    \begin{lrbox}{\dummybox} 
    \begin{minipage}[t]{.9\textwidth}
 }
 {  \end{minipage} \end{lrbox}
      \raisebox{-\depth}{\hspace{.5in}\hspace{-4mm}
       \psshadowbox[fillstyle=solid,fillcolor=hcolor,framesep=8mm]
    {\usebox{\dummybox}}} \end{center}
     \vspace{2em}
 }

%%%%%%%%%%%%%%%%%%%%%%%%%%%%%%%%%%%%%%%%%%%%%%%%%%%%
%%%                Header no shadow              %%%
%%%%%%%%%%%%%%%%%%%%%%%%%%%%%%%%%%%%%%%%%%%%%%%%%%%%
\newenvironment{headerns}[1][1. 1. 1.] {
    \vspace{2em}

    \begin{center}
    \newrgbcolor{hcolor}{#1}
    \begin{lrbox}{\dummybox} 
    \begin{minipage}[t]{.9\textwidth}
 }
 {  \end{minipage} \end{lrbox}
      \raisebox{-\depth}{\hspace{.5in}\hspace{-4mm}
       \psshadowbox[fillstyle=solid,fillcolor=hcolor,framesep=8mm,shadowsize=0pt]
    {\usebox{\dummybox}}} \end{center}
     \vspace{2em}
 }

%%%%%%%%%%%%%%%%%%%%%%%%%%%%%%%%%%%%%%%%%%%%%%%%%%%%
%%%                Poster                        %%%
%%%%%%%%%%%%%%%%%%%%%%%%%%%%%%%%%%%%%%%%%%%%%%%%%%%%
\newsavebox{\dummybox}
\newsavebox{\spalten}

\newenvironment{poster} {
   \vfill \begin{minipage}{\textwidth}  }
  {\end{minipage} \vfill}

%%%%%%%%%%%%%%%%%%%%%%%%%%%%%%%%%%%%%%%%%%%%%%%%%%%%
%%%                pcolumn                       %%%
%%%%%%%%%%%%%%%%%%%%%%%%%%%%%%%%%%%%%%%%%%%%%%%%%%%%
\newcommand{\columnfrac}{.3}
\newenvironment{pcolumn}{%
  \hfill\begin{minipage}[t]{\columnfrac\textwidth}}{\end{minipage}\hfill%
}

%%%%%%%%%%%%%%%%%%%%%%%%%%%%%%%%%%%%%%%%%%%%%%%%%%%%
%%%                pbox                          %%%
%%%%%%%%%%%%%%%%%%%%%%%%%%%%%%%%%%%%%%%%%%%%%%%%%%%%
\newenvironment{pbox}[1][1. 1. 1.] {
   \newrgbcolor{bcolor}{#1}
  \begin{lrbox}{\dummybox}%
    \begin{minipage}{0.96\linewidth}}%
    {\end{minipage}
  \end{lrbox}
  \raisebox{-\depth}{\hspace{-.5em} 
   \psshadowbox[fillstyle=solid,fillcolor=bcolor,framesep=1em]{\usebox{\dummybox}}} \vspace{2em} \vfill}

%%%%%%%%%%%%%%%%%%%%%%%%%%%%%%%%%%%%%%%%%%%%%%%%%%%%
%%%                pbox   no shadow              %%%
%%%%%%%%%%%%%%%%%%%%%%%%%%%%%%%%%%%%%%%%%%%%%%%%%%%%
\newenvironment{pboxns}[1][1. 1. 1.] {
   \newrgbcolor{bcolor}{#1}
  \begin{lrbox}{\dummybox}%
    \begin{minipage}{0.96\linewidth}}%
    {\end{minipage}
  \end{lrbox}
  \raisebox{-\depth}{\hspace{-.5em} 
   \psshadowbox[fillstyle=solid,fillcolor=bcolor,framesep=1em,shadowsize=0pt]{\usebox{\dummybox}}} \vspace{2em} \vfill}




\begin{document}
\lstset{language=PHP,showstringspaces=false}

\renewcommand{\columnfrac}{.45}
\begin{header}
\begin{center}

\VeryHuge
AspectPHP\\
\huge
An Extension for Aspect-Oriented Programming to the PHP Programming Language\\
\LARGE
Sebastian Bergmann $<$sb@sebastian-bergmann.de$>$
\end{center}
\end{header}

\begin{poster}

\begin{pcolumn}

\begin{pbox}
\large
  \section{Introduction}

    \subsection{PHP}

      PHP [1] is a widely-used general-purpose programming language that is
      especially suited for Web development. It is dynamically typed and
      supports both procedural and object-oriented programming.
      For the latter it provides a class-based object model that is similar to
      those of C\# or Java.

    \subsection{AspectPHP}

      The goal of the AspectPHP [2] project is to provide an extension to
      the PHP programming language that supports aspect-oriented programming
      using dynamic weaving of aspect- and base-code at runtime.

  \section{Characteristics of AspectPHP}

    AspectPHP can be characterized as follows:

    \begin{enumerate}

      \item

        Possible Quantifications (Join Points): Attribute read and write access,
        call and execution of a method, execution of a catch-block,
        initialization of an object.

      \item

         Possible Interactions

        \begin{enumerate}

          \item

            Advice: before-, after- and around-advices for the above mentioned
            join points.

          \item

            Introduction: Attributes and methods can be added to existing PHP
            classes, inheritance relations and interface implementations of
            existing PHP classes can be altered.

        \end{enumerate}

    \end{enumerate}

  \section{Implementation}

    \subsection{Possible Implementation Approaches}

      An extension to the PHP programming language that supports aspect-oriented
      programming using dynamic weaving should be implemented in one of two ways:
      The Zend Engine, the compiler and interpreter of PHP that is written in C,
      can be extended. Alternatively, the language extension can also be written
      in the PHP programming language itself by leveraging its meta-programming
      capabilites.

    \subsection{Implementation of AspectPHP}

      The current prototype for AspectPHP is implemented in the PHP programming
      language itself. It uses the Runkit [3] extension to the PHP interpreter to
      manipulate the PHP-internal bytecode of classes and methods.\\
      \\
      Aspects are plain PHP classes that contain pointcut, advice, or
      introduction annotations in their code comments.\\
      \\
      AspectPHP uses a classloader that handles the loading of the source
      files that contain the code for aspects and classes.
      When loading the source file for an aspect, the class representing the
      aspect is searched for annotations and the pointcuts declared in the
      aspect are registered.
      When loading the source file for a class, the necessary hooks for the
      join points as well as introductions declared by aspects are inserted
      into the PHP-internal bytecode of the class and its methods.\\
      \\
      At runtime, the previously introduced hooks for the join points check
      whether or not an aspect has a pointcut with an advice for the current
      join point. If that is the case, AspectPHP's Dispatcher class handles
      the execution of the corresponding advice and passes the current context
      to the advice's method(s).\\
      The variable \lstinline{$joinPoint} is available in the advice methods and
      contains context information for the current join point. This information
      includes, for instance, the calling object as well as the object called for
      a method call join point.

\end{pbox}

\end{pcolumn}
\begin{pcolumn}

\begin{pbox}
\section{Example: Logging Method Calls}

\begin{lstlisting}
<?php
/**
 * Declare a new pointcut named callPointCut using the @pointcut annotation that captures
 * join points on a method regardless of modifier, class, method, or number of parameters.
 * @pointcut callPointCut() : call(* *->*(..));
 *
 * Bind the after() method of this aspect as an after-advice to the
 * previously declared pointcut.
 * @after callPointCut : LoggingAspect->after();
 */
class LoggingAspect {
    public function after($joinPoint) {
        printf(
          "%s->%s() called %s->%s()\n",
          $joinPoint->getSource()->getDeclaringClass()->getName(),
          $joinPoint->getSource()->getName(),
          $joinPoint->getTarget()->getDeclaringClass()->getName(),
          $joinPoint->getTarget()->getName()
        );
    }
}
?>
\end{lstlisting}

\section{Related Work}
\large
  The author of this poster knows of four existing approaches to faciliate
  aspect-oriented programming with the PHP programming language:

  \begin{itemize}

    \item

      \emph{PHPAspect} [4] uses a compiler, written in the PHP programming
      language, that performs static weaving using source code transformations.
      A downside of this approach is that advantages that stem from PHP's
      interpreted nature are lost.

    \item

      \emph{Aspect-Oriented PHP} [5] uses a preprocessor for the PHP programming
      language written in Java that is responsible for the weaving of aspect- and
      base-code. Due to its Java implementation this approach does not integrate
      seamlessly with the PHP platform.

    \item

      \emph{aspectPHP} [6] is a reimplementation of Aspect-Oriented PHP in C,
      available as a patch against (not as an extension to) PHP 4.3.10.

    \item

      The \emph{AOP Library for PHP} [7] requires manual changes to the base-code
      and thus does not provide obliviousness.

  \end{itemize}

\section{About the Author}
Sebastian Bergmann is a Computer Science student in his final year at the University of Bonn.
He spends his free time with the development of Free Software, is a member of the PHP and Gentoo Linux development teams and author of a variety of PHP software projects such as PHPUnit.

\section{References}
\normalsize
\begin{enumerate}
  \item http://www.php.net/
  \item http://www.sebastian-bergmann.de/AspectPHP/
  \item http://pecl.php.net/package/runkit
  \item http://www.phpaspect.org/
  \item http://www.aophp.net/
  \item http://www.cs.toronto.edu/\textasciitilde yijun/aspectPHP/
  \item http://pear.php.net/pepr/pepr-proposal-show.php?id=315
\end{enumerate}
\end{pbox}
\end{pcolumn}
\end{poster}
\end{document}
